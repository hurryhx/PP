\documentclass{article}
\usepackage{fontspec}
\usepackage{zhspacing,url,amsmath,amssymb,verbatim}
\zhspacing
\usepackage{listings}
\usepackage[hyperfootnotes=false,colorlinks,linkcolor=blue,anchorcolor=blue,citecolor=blue]{hyperref}
\usepackage[sorting=none]{biblatex}
%\usepackage[dvips]{graphicx}
\usepackage{minted}
\usepackage{subfigure}
\usepackage{indentfirst}
\usepackage{cases}
\usepackage{environ}
\usepackage{array}
\usepackage{graphicx}
\usepackage[top=1in, bottom=1in, left=1.25in, right=1.25in]{geometry}
%\usepackage{tikz}
%\usepackage{dot2texi}

\newcommand{\inputmintedConfigured}[3][]{\inputminted[fontsize=\footnotesize,
	label=#3,linenos,frame=lines,framesep=0.8em,tabsize=4,#1]{#2}{#3}}

\newcommand{\txtsrc}[2][]{\inputmintedConfigured[#1]{text}{#2}}
\newcommand{\txtsrcpart}[4][]{\txtsrc[firstline=#3,firstnumber=#3,lastline=#4,#1]{#2}}

\newcommand{\cppsrc}[2][]{\inputmintedConfigured[#1]{cpp}{#2}}
\newcommand{\cppsrcpart}[4][]{\cppsrc[firstline=#3,firstnumber=#3,lastline=#4,#1]{#2}}


\newcommand{\figref}[1]{\hyperref[fig:#1]{Figure\ref*{fig:#1}}}
\newcommand{\tableref}[1]{\hyperref[table:#1]{Table\ref*{table:#1}}}
\newcommand{\centerize}[1]{\begin{center} #1 \end{center}}

\newcommand{\cmd}[1]{{\it #1}}
\newcommand{\ccmd}[1]{\centerize{\cmd{#1}}}

\newcommand{\proj}{River and Frog Arcade Game}

\title{\proj ~using pthread}
\author{Xin Huang\\ Dept. of CST, THU\\ ID: 2011011253}
\date{\today}

\addbibresource{refs.bib}
\begin{document}
\maketitle

\begin{abstract}
	A \proj ~implemented with gtkmm and pthread.

	This is {\it homework 2} for course {\it Parallel Programming}

	{\bf Keyword} \proj, pthread
\end{abstract}

\tableofcontents

\clearpage

\section{Instruction}
	\large
	\subsection{Prerequisite}
	This program uses {\it gtkmm-3.0}(along with glib and gdkmm) and {\it pthread} as its back end, so you {\bf MUST} have
		them installed.
	\subsection{Compilation}
		Invoke \cmd{make} to compile the source code.  Also you can
		set environment variable \cmd{DEFINES="-DUSE\_PTHREAD"} to enable pthread functionality.
		Executable is sorted in \cmd{bin/river-and-frog}
	\subsection{Execution}
		Invoke \ccmd{make run} to run with the default setting.

		For advanced options, see section~\ref{sec:option}
		You can stretch the window for bigger size.

\section{Introduction}
	You are to control a frog jump over the river
	by stepping on wood floating in the river. The frog dies if it jumps into
	the river(maybe crocodiles waiting with hunger).

\section{Control Setting}
	\begin{itemize}
		\item
			In single mode, Player 1 control the frog using arrow keys.
		\item
			In two player mode, Player 1 use arrow keys, and player two use \cmd{w, s, a, d} keys.
	\end{itemize}

\section{Programming Design}
	\begin{itemize}
	\large
		\item
			The program is written with Gtkmm and pthread. This program has
			Game, Layer, Wood and Frog. Game deals with the start and update.
			Layer calculates the speed and position of each wood on the layer.
			Wood loads the image and grasp the position of the frog(is the
			frog on the wood). Frog works for the movement of the frog.
		\item
			I use Glib::timeout to refresh the Game and the canvas. Game is
			the conductor. Every time the Game::update executes the speed and
			position of woods and frogs will be computed and update the canvas.
		\item
			Multithread programming is used in Layer. The number of threads
			is same to the number of layers. Each thread computes the collision
			on its own. When all collision computation finished(use
			pthread join to block), the update function will move the wood
			and the frog. And then the update loop is finished.
		\item
			Game controls Layer, Layer controls Wood, Wood controls Frog.
		\item
			Game::start() first invoke Game::init() first to initialize the canvas,  woods
			and the frogs. Then invoke sigc::mem_fun and Glib::signal_timeout
			to refresh the Game::update(). The Game::update() use a timer to
			count the remain time to let the Layer calculate the every loop's
			time.
		\item
			Layer::update() is run in each thread. Every thread will get a
			layer and calculate the woods collisions on its layer. The layer
			does the calculation and the movement of the woods on the layer.
		\item
			Each time the Layer would move only one wood that will immediately
			collides with another. The collision form is elastic collision.
			When the remain time is run out, the thread will wait for other
			threads to stop.
		\item
			When Layer::update() finishes the Game::update() will move the frog.
			The frog can move to up, down, left, right. The message is deal with
			Gdk Keyval.
		\item
			The frog will go up to the next layer if user presses left key.
			The Frog object will be removed from the current Wood and jump to
			(add to) the upper Wood. But if there is no woods, the frog will be
			removed and added to the basement.

	\end{itemize}

\section{Features}
	\large
	\begin{itemize}
		\item
			River consists of multiple layers, wood slabs can only move in one layer.
		\item
			Wood slabs have masses proportional to its width.

		\item
			By default setting, when two woods hit, the resulting speed of
			the wood slabs obeys the conservation of momentum and energy.

			Watch out if a large piece of wood slab's about to hit you!
		\item
			Wood slabs are in different size and moving in different speed.
			The wood slabs in last layer moves really quick, so be careful
			when you move
		\item
			Frog can jump between wood slabs in two different layers, but
			can not change to wood slab in the same layer.
		\item
			The upper the wood slab, the faster it will be.
		\item
			There is a slight acceleration in frogs moving speed, and frog
			will stop as soon as you release the control keys.

			Get across the river as fast as you can!
	\end{itemize}

\section{Specification of Pthread in Program}
	Pthread is used for calculate collisions between wood slabs.
	During each frame, each layer is assigned a thread to calculate
	collisions simultaneous.

\section{Experience}
	Writing a GUI program with Gtkmm is not so convenient. I wrote the program in the first
	time, in which I need to look up the documentations and ask classmates for help. When
	I have debug it for two day and nothing changed, I rewrite the program. Though the
	experience is tough and not very pleasant, I now know how to write a Gtk program with
	pthread. And now I realize it's also important to schedule the time. In the first place
	I want to write a n-body simulation program and switch to this task, but it's too hard
	and I waste two days. Fortunately I change my idea and I can finally finish this task.
	Preparing and scheduling are now have to be taken in my consideration.

\section{Screenshots}
\newcommand{\scrnshts}[3]{
	\begin{figure}[!ht]
		\centering
		\includegraphics[scale=#3]{#2}
		\caption{#1}
	\end{figure}
}

\scrnshts{Game start}{../res/1.png}{0.6}
\scrnshts{During game}{../res/2.png}{0.6}
\scrnshts{Game with two players}{../res/3.png}{0.6}

\clearpage
\printbibliography

\end{document}

